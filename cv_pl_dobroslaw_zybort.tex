%% start of file `cv_pl_dobroslaw_zybort2.tex'.
%% Copyright 2013 Dobrosław Żybort (dobroslaw.zybort@gmail.com).


\documentclass[11pt,a4paper,sans,polish]{moderncv}   % possible options include font size ('10pt', '11pt' and '12pt'), paper size ('a4paper', 'letterpaper', 'a5paper', 'legalpaper', 'executivepaper' and 'landscape') and font family ('sans' and 'roman')

\usepackage[T1]{fontenc}
\usepackage[a4paper]{geometry}
\geometry{verbose,tmargin=1.5cm,bmargin=2cm,lmargin=0.07\paperwidth,rmargin=0.07\paperwidth,footskip=1cm}

\usepackage{fancyhdr}
\pagestyle{fancy}
\setcounter{secnumdepth}{2}
\setcounter{tocdepth}{2}
\setlength{\parskip}{\medskipamount}
\setlength{\parindent}{0pt}
\usepackage{babel}
\usepackage{setspace}
\onehalfspacing
\ifx\hypersetup\undefined
  \AtBeginDocument{%
    \hypersetup{unicode=true,
 bookmarks=true,bookmarksnumbered=true,bookmarksopen=true,bookmarksopenlevel=1,
 breaklinks=false,pdfborder={0 0 0},backref=false,colorlinks=false,
 pdfauthor={Dobrosław Żybort},
 pdfkeywords={Dobrosław Żybort,Curriculum Vitae,Resume},
 pdfpagelayout=OneColumn, pdfnewwindow=true, pdfstartview=XYZ, plainpages=false}
  }
\else
  \hypersetup{unicode=true,
 bookmarks=true,bookmarksnumbered=true,bookmarksopen=true,bookmarksopenlevel=1,
 breaklinks=false,pdfborder={0 0 0},backref=false,colorlinks=false,
 pdfauthor={Dobrosław Żybort},
 pdfkeywords={Dobrosław Żybort,Curriculum Vitae,Resume},
 pdfpagelayout=OneColumn, pdfnewwindow=true, pdfstartview=XYZ, plainpages=false}
\fi

% moderncv themes
\moderncvstyle{classic}                      % style options are 'casual' (default), 'classic', 'oldstyle' and 'banking'
\moderncvcolor{blue}                         % color options 'blue' (default), 'orange', 'green', 'red', 'purple', 'grey' and 'black'
%\renewcommand{\familydefault}{\sfdefault}   % to set the default font; use '\sfdefault' for the default sans serif font, '\rmdefault' for the default roman one, or any tex font name
\nopagenumbers{}                             % uncomment to suppress automatic page numbering for CVs longer than one page

% character encoding
\usepackage[utf8x]{inputenc}                 % if you are not using xelatex ou lualatex, replace by the encoding you are using

% adjust the page margins
% \usepackage[scale=0.75]{geometry}
\setlength{\hintscolumnwidth}{3cm}           % if you want to change the width of the column with the dates
%\setlength{\makecvtitlenamewidth}{10cm}     % for the 'classic' style, if you want to force the width allocated to your name and avoid line breaks. be careful though, the length is normally calculated to avoid any overlap with your personal info; use this at your own typographical risks...

% personal data
\firstname{Dobrosław}
\familyname{Żybort}
\title{Curriculum Vit\ae{}}
\address{Broniewskiego 26a/m.12}{93-142 Łódź}{Polska}
\mobile{791 673 189}
% \phone{+2~(345)~678~901}
% \fax{+3~(456)~789~012}
\email{dobroslaw.zybort@gmail.com}
\homepage{https://bitbucket.org/matrixik/}
% \extrainfo{additional information}
% \photo[64pt][0.4pt]{picture}               % '64pt' is the height the picture must be resized to, 0.4pt is the thickness of the frame around it (put it to 0pt for no frame) and 'picture' is the name of the picture file
% \quote{Some quote}

% Wyrażenie zgody na przetwarzanie danych osobowych
\fancypagestyle{zgoda}{%
  \fancyhf{}
  \renewcommand\headrulewidth{0pt}
  \fancyfoot[R]{\fontsize{9}{11} \selectfont Wyrażam zgodę na przetwarzanie moich danych osobowych w celach rekrutacji, zgodnie z Ustawą z dn. 29 sierpnia 1997 r. o ochronie danych osobowych (Dz. U. z 2002 r. Nr 101, poz. 926 z późniejszymi zmianami).}
}


%% Styl

% Wielkość czcionki
% \renewcommand*{\namefont}{\fontsize{34}{36}\mdseries\upshape}
% \renewcommand*{\titlefont}{\LARGE\mdseries\slshape}
% \renewcommand*{\addressfont}{\small\mdseries\slshape}
% \renewcommand*{\quotefont}{\large\slshape}
% \renewcommand*{\sectionfont}{\Large\mdseries\upshape}
% \renewcommand*{\subsectionfont}{\large\mdseries\upshape}
% \renewcommand*{\hintfont}{}

\renewcommand*{\sectionfont}{\fontsize{14}{16} \mdseries\upshape}

% Wygląd poziomej linii
%\renewcommand*{\section}[1]{%
%  \par\addvspace{1.5ex}%
%  \phantomsection{}% reset the anchor for hyperrefs
%  \addcontentsline{toc}{section}{#1}%
%  \parbox[t]{\hintscolumnwidth}{\strut\raggedleft\raisebox{\baseletterheight}{\color{color1}\rule{\hintscolumnwidth}{0.35ex}}}%
%  \hspace{\separatorcolumnwidth}%
%  \parbox[t]{\maincolumnwidth}{\strut\sectionstyle{#1}}%
%  \par\nobreak\addvspace{1ex}\@afterheading}% to avoid a pagebreak after the heading

%----------------------------------------------------------------------------------
%            content
%----------------------------------------------------------------------------------
\begin{document}
%-----       resume       ---------------------------------------------------------
\makecvtitle

\vspace{-0.7cm}


\section{Wykształcenie}

\cventry{\small2005 -- obecnie}{Wydział Elektrotechniki, Elektroniki, Informatyki
i Automatyki}{Politechnika Łódzka}{Łódź}{Polska}{}
\vspace{-0.2cm}

\cvitem{Specjalność}{\textbf{Bazy Danych i Systemy Ekspertowe}}


\section{Doświadczenie}

\cventry{\small2012-12 -- obecnie}
	{oauth2, ListDict, conf i inne}{open source}{}
	{https://bitbucket.org/gosimple/}
	{Biblioteki napisane w języku programowania Go}

\cventry{\small2012-07 -- obecnie}
	{Wordpress autoposter}{projekt prywatny}{}
	{kod dostępny na~życzenie}
	{Napisany w języku Python bot automatyzujący wysyłanie postów na platformę Wordpress}

\cventry{\small2012-11 -- 2013-01}
	{PastebinGo}{open source}{}
	{https://bitbucket.org/matrixik/pastebingo/}
	{Strona typu pastebin napisana w języku Go}

\cventry{\small2012-01}
	{Studio Tam-Tam}{Łódź}{Polska}
	{}
	{Praktyka zawodowa w zakresie obsługi informatycznej firmy}

\cventry{\small2009 -- obecnie}
	{Betfair Watcher, Engine Checker i inne}{open source}{}
	{https://bitbucket.org/matrixik/}
	{Głównie napisane w języku Python i Go boty oraz programy użytkowe}

\cventry{\small2008-07 -- 2009-01}
	{Tekka Sp. z o.o.}{Łódź}{Polska}
	{Tester oprogramowania}
	{Testowanie oraz tworzenie dokumentacji dla systemu do zarządzania
	zleceniami w firmie logistycznej}


\section{Umiejętności}

\cvitemwithcomment
	{Język angielski}
	{Dobry}{Znajomość terminologii technicznej}

\cvitemwithcomment
	{Języki prog.}
	{Python, Go (młody, dynamicznie rozwijający się język programowania)}{}

\cvitemwithcomment
	{Bazy danych}
	{SQL Server, Oracle}{}

\cvitemwithcomment
	{Web}
	{HTML, CSS}{}

\cvitemwithcomment
	{Systemy}
	{Windows XP/7, Linux Debian}{}

\cvitemwithcomment
	{Edytory}
	{Microsoft Office, LyX - edytor WYSIWYM (\LaTeX{})}{}


\section{Dodatkowe umiejętności}

\cvitem{}{Prawo jazdy kat. B}{}


\section{Zainteresowania}

\cvitem{}{Tworzenie botów oraz web scraper-ów, wycieczki piesze, manga}{}


\thispagestyle{zgoda}

\end{document}
